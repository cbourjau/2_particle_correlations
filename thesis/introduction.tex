\chapter{Introduction}
\label{sec:intro}

%Why is particle physics great and fascinating
% - Write about idea not experiments yet
% Study of everything - but on the smallest possible length and time scale
% Fundamental, undividible  building blocks of matter and there interplay
% Is always and everywhere in the universe valid
% Also at the very begining of the universe, microseconds after the big bang
Elementary particle physics is the study of everything. While this statement might seem rather shallow and fishing for attention (which it is), it is still true in many more ways than one would initially think. Elementary particle physics studies the fundamental building blocks of all the types of matter ever observed. It is valid here in Lund, Sweden just as well as at the most distant outskirts of the universe - and it is believed to always have been and always will continue to do so. Even split seconds after the Big Bang the laws of particle physics governed the properties of this extremely hot and dens soup of matter evolving into the vast variety of particles known today.

% Cross over to measurments
However, many of the fundamental laws are not yet understood or even discovered, despite decades of work by thousands of physicists. During this time, larger and larger particle accelerators were built reproducing the state of matter at ever decreasing time scales after the Big Bang. 
% CERN, LHC, ALICE, Astrophysics
% Many possible fields to study but cut rather straight to the matter at extreme densities and temperatures called a quark gluon plasma

The current record holder in this quest is CERN with its \emph{Large Hadron Collider} (LHC), being able to collide particles at unprecedented energies and frequencies. The LHC houses four experiments each having a specific scope. The one focused on studying matter at extreme conditions such as the Big Bang, which is created when colliding lead nuclei (PbPb), is the ALICE experiment. When the first results of such collisions became available in 2011 they appeared  to be in agreement with previous studies predicting that PbPb collisions at the LHC will be able to create a state of matter known as the \emph{Quark Gluon Plasma} (QGP).
% Well understood , but big surprise when looking at pPb to find a baseline!
Intriguingly, conducting studies on proton-lead (pPb) collisions yielded similar results, though the creation of the QGP was deemed impossible in this collision system.
% New theories emerged, as is tradition in particle physics, and it is up to the experimentalists to probe; see lene; keep it reasonable general to fit next point
Since then, several new or refined models were proposed to explain these experimental observations.
% Goal of this thesis
% do the above probing for cgc, hydro, and color reconnection
Thus, the ball is back in the experimentalist's court to gather evidence for which theories hold, which need to be refined and which should be discarded.

This thesis presents the methods and findings from investigating so called \emph{hard} pPb collisions by means of two-particle angular correlations. Subsequently, the yielded results are compared to the anticipated results of three possible models explaining the state of matter created at pPb collisions.

The goal of this thesis was to provide new insights into the structure of proton-nucleus collisions and thus contributing to ultimately ruling out some, or confirming other models.

% What was "literally done' in this thesis?
The work conducted during this project included a complete implementation\footnote{The complete source code excluding non-public data files may be found at \cite{Bourjau2014}} of the two-particle correlations method described in chapter \ref{chap:methods}. Furthermore, considerable time and effort was spend on further studies regarding the autocorrelations introduced by detector deficiencies, particle densities and detector acceptances. These studies led to a novel, less model dependent and less complex efficiency correction method for two-particle correlation studies.

The following chapter introduces the main concepts of the Standard Model before summarizing important results of heavy ion physics. A focus is then placed on the observation of the so-called \emph{flow-like} phenomena in pPb collisions. Three models that can possibly explain this effect are introduced and their anticipated responses to a hardened event sample are discussed. Chapter \ref{chap:alice} describes the LHC and the ALICE experiment as well as the detectors of interest to the carried out analysis. The fourth chapter first covers the event and track selection criteria before describing the extraction and correction of two-particle correlation from the event sample. Subsequently, the \emph{subtraction} method revealing the flow like phenomenon is discussed and the applied methods are tested for Monte Carlo closure. Results are presented in chapter \ref{chap:results} while their discussion takes place in chapter \ref{chap:discussion}. Finally, a summary and outlook is given in the last chapter.


% Comment on structure

%%% Local Variables: 
%%% mode: latex
%%% TeX-master: "main"
%%% End: 