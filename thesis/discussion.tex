
\chapter{Discussion}
\label{chap:discussion}
%%%%%%%%%%
% No recap of what or how something was measured etc!!!!
%%%%%%%%%%
The results shown in chap.~\ref{chap:results} had two primary objectives. The first being the reproduction of the published results while applying the novel average-efficiency method (cf. sec.~\ref{sec:single-value-correction}). This was achieved for all investigated intervals of \ptassoc and \pttrig with respect to the ridge yields and base line of uncorrelated particles taking into account the fact that the published results were erroneously lowered by approximately $5\%$ (cf. sec. \ref{sec:no-thresh}). Thus, the implemented methods can be regarded as a well working foundation for the studies on the hardened event sample, but one further consideration has to be made before discussing the introduction of a threshold:\\
The motivation for the definition of the subtraction method was the assumption that the di-jet contributions were independent of the multiplicity class and would thus cancel each other, revealing only non-di-jet related differences. However, a remaining peak on the near side of the double ridge was observed after applying the subtraction in fig.~\ref{fig:comparison_to_pub} and fig.~\ref{fig:subtractions_high_pt}. This suggests that the jet contribution on the near-side are indeed not identical in the two multiplicity classes. Due to momentum conservation, this will also cause the recoils of the jets (the away side part of the di-jet structure) to be of different magnitude. However, this cannot be verified since the recoil coincides with the so far unexplained away side of the double-ridge. In the no-threshold cases, the mismatch between the two jet peaks, and thus their recoil was small in comparison to the total ridge size and could therefore be disregarded. However, the conclusion from this discussion is that the peak-region as well as the away side may include considerable di-jet contributions, if one of the centrality classes becomes biased towards jets as discussed below. However, it is important to note that the long-range near side is not affected by di-jets and thus represents an unbiased part of the double-ridge.\\

Fig.~\ref{fig:y_px_thresh} (bottom) displays the dependence of $Y(\dphi)$ on \ptthresh for each multiplicity class relative to the no threshold case. The general baseline increase of all four classes can be attributed to the dependence of the number of charged particles $N_{ch}$ to the  $\left< \pt \right>$ (which is raised by the threshold) of the underlying event (cf. fig.~\ref{fig:cf_comparison}) which is a well understood effect.\\

The enhancement of the away-side peak, on the other hand, is less clear. Events of the high multiplicity class ($0-20\%$) may be expected to include several jets per event. The requirement of a high \pt particle will thus be met by a large subset of the initial event sample. On the other hand, the same does not apply to the low multiplicity class ($60-100\%$): Events meeting the threshold criteria are likely to have only one di-jet which also includes its high \pt particle. Furthermore, only a small fraction of the original event sample will be available for further processing.\\
Hence, the low multiplicity event sample is expected to exhibit a strong bias towards di-jet events due to the threshold while the high multiplicity class is expected to be less enriched by hard events in comparison.\\

Together with the initial discussion about the possible mismatches of the di-jet contributions in the high and low multiplicity classes, this explains the dip on the near-side and negative ridge on the away-side seen when applying the subtraction method to MC-truth and MC-reconstructed data while requiring a threshold particle as shown in fig. \ref{fig:subtraction_MC}.\\

Applying the subtraction method to experimental data, as shown in fig.~\ref{fig:subtractions_high_pt}, combines the above di-jet bias with the flow-like effect which is not included in the MC data. In this combination it appears as if the away-side vanishes completely at \SI{4}{GeV/c} and above. It is intriguing that these two processes cancel each other so precisely and was, in fact, the very reason for the further investigation of the threshold effect. However, as of yet no convincing explanation for a connection between these two effects has been found.

At this point, it is clear that the threshold breaks the assumption that the di-jet contribution of the low multiplicity class cancels the one in the high multiplicity class. Hence, the long-range near-side is now the primary region for studying the evolution of the double ridge structure in hard events. However, a legitimate conclusion to this point is only possible if approximate MC closure exists for this region. Sec.~\ref{sec:closure_with_thresh} addressed this question but a further discussion about the long-range near-side will follow here. Fig.~\ref{fig:closure_structure_w_threshold} shows that closure is achieved for the $0-20\%$ centrality class while a di-jet-like structure of non-closure is present for $60-100\%$. It should be noted that no direct connection between this structure and the above described bias towards di-jet events in low multiplicity collisions is found as of yet. The region of interest is, however, the long-range near side. The \deta projection suggests that this region is in on the same level as the remaining baseline which in turn coincides with the one of the high multiplicity class. This is strongly suggestive that the here applied method did not systematically alter the long-range near-side.\\

Fig.~\ref{fig:ridge_yield_evo} depicts the evolution of the ridge yields with increasing threshold and with systematic uncertainties from the subtraction of the MC-truth and MC-reconstructed results. This diminishes the significance of the away-side. The peak yield does not suffer such a large systematical error in this definition but has to be considered carefully due to the non-closure in this region. Neither of theses sources of uncertainty are present for the long-range near-side. No dependence of this observable on \ptthresh was found. This might indicate that the process causing the flow-like effect is indeed not sensitive to the hardening of the event sample which is the anticipated behavior of the  \gls{cgc} and hydrodynamic-expansion model but not of the \gls{cr} for which deviations were expected (cf. sec.~\ref{sec:hard-soft}). This is, however, not sufficient to rule out \gls{cr} yet. Detailed calculation for \gls{cr} in \gls{pPb} are still scarce, especially under the here investigated conditions, and may yield results in accordance to the here presented results when conducted.



%%% Local Variables: 
%%% mode: latex
%%% TeX-master: "main"
%%% End: 
