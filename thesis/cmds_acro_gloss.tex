\newcommand{\pt}{\ensuremath{p_{\text{T}}} }
\newcommand{\ptbold}{\boldmath{$p_{\text{T}}$} }
\newcommand{\ptassoc}{\ensuremath{p_{\text{T}}^{\text{asso}}} }
\newcommand{\pttrig}{\ensuremath{p_{\text{T}}^{\text{trig}}} }
\newcommand{\ptthresh}{\ensuremath{p_{\text{T}}^{\text{thresh}}} }
\newcommand{\zvtx}{\ensuremath{z_{\text{vtx}}} }
\newcommand{\deta}{\ensuremath{\Delta\eta} }
\newcommand{\dphi}{\ensuremath{\Delta\varphi} }
\newcommand{\Yrecon}{\ensuremath{Y^{\text{recon}}} }
\newcommand{\Ytruth}{\ensuremath{Y^{\text{truth}}} }
\newcommand{\B}{\ensuremath{B(\deta, \dphi)} }
\newcommand{\Y}{\ensuremath{Y(\deta, \dphi)} }
\newcommand{\Sig}{\ensuremath{S(\deta, \dphi)} }

%%%%%% Acronyms and glossary %%%%%%%%%%%

\newacronym{alice}{ALICE}{A Large Ion Collider Experiment}
\newacronym{atlas}{ATLAS}{A Toroidal LHC Apparatus}
\newacronym{cms}{CMS}{Compact Muon Solenoid}
\newacronym{lhcb}{LHCb}{Large Hadron Collider beauty}
\newacronym{cern}{CERN}{Conseil Européen pour la Recherche Nucléaire}
\newacronym{lhc}{LHC}{Large Hadron Collider}
\newacronym{lep}{LEP}{Large Electron–Positron Collider}
\newacronym{qcd}{QCD}{Quantum Chromodynamics}
\newacronym{qed}{QED}{Quantum Electrodynamics}
\newacronym{qgp}{QGP}{Quark Gluon Plasma}
\newacronym{cgc}{CGC}{Color Glass Condensate}
\newacronym{cr}{CR}{Color Reconnection}
\newacronym{sm}{SM}{Standard Model}
\newacronym{pt}{\pt}{Transverse momentum}
\newacronym{y}{\ensuremath{y} }{Pseudo rapidity}
\newacronym{dca}{DCA}{Distance of Closest Approach}
\newacronym{Pb}{Pb}{lead}

\newacronym{pp}{pp}{proton-proton}
\newacronym{PbPb}{PbPb}{lead-lead}
\newacronym{AA}{AA}{nucleus-nucleus}
\newacronym{pPb}{pPb}{proton-lead}
\newacronym{pA}{pA}{proton-nucleus}

\newacronym{mc}{MC}{Monte Carlo}

\newacronym{pid}{PID}{Particle Identification}
\newacronym{tof}{TOF}{time-of-flight}
\newacronym{its}{ITS}{Inner Tracking System}
\newacronym{tpc}{TPC}{Time Projection Chamber}
\newacronym{spd}{SPD}{Silicon Pixel Detectors}
\newacronym{sdd}{SDD}{Silicon Drift Detectors}
\newacronym{ssd}{SSD}{Silicon mico-Strip Detectors}
\newacronym{mwpc}{MWPC}{Multi Wire Proportionality Chamber}



\newacronym{eta}{pseudo rapidity $\eta$}{Spatial coordinate describing the angle to the beam axis. It is a approximation for of rapidity $y$ which may be computed without knowledge of the particle's energy. It is defined as $\eta = -\ln\left[\tan\left(\frac{\theta}{2}\right)\right]$.}

\newglossaryentry{near-side}{
  name={near-side},
  description={The near-side describes the area of \dphi marked by the jet peak, usually set around $\dphi \approx 0$}
}
\newglossaryentry{away-side}{
  name={away-side},
  description={The away-side describes the area of \dphi dominated by the recoil of the jet peak, hence usually set around $\dphi \approx \pi$}
}
\newglossaryentry{soft}{
  name={soft},
  description={Soft events are defined as events with no track above a given threshold \ptthresh}
}
\newglossaryentry{hard}{
  name={hard},
  description={Hard events are defined as events with at least one track above a given threshold \ptthresh}
}

\newglossaryentry{golden}{
  name={golden cuts},
  description={Event cut}
}

\newglossaryentry{TPC-only}{name=TPC-only, description=Event cut}
\newglossaryentry{di-jet}{name=di-jet, description=A di-jet consists of a narrow jet and a broader recoil on the opposit site. In a correlation plot of \deta \dphi it exhibits a jet peak on the near side and a recoil ridge on the away side}
\newglossaryentry{double-ridge}{
        name={double-ridge},
        description={Term commonly used to describe the structure remaining when subtracting the total associated yield of low multiplicity events from the one of high multiplicity events; Two ridge like structures along \deta located at $\dphi\approx 0$ and $\dphi \approx \pi$.}}

\newglossaryentry{DPMJET}{
  name={Dual Parton Model Jet},
  description={Event cut}
}

\makeglossaries

%%% Local Variables: 
%%% mode: plain-tex
%%% TeX-master: "main"
%%% End: 