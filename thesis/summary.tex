
\chapter{Summary}
\label{chap:summary}

% Motivation
The unanticipated, previously reported \cite{Abelev2012,CMSCollaborationChatrchyan2013}, flow-like effects in high multiplicity \gls{pPb} collisions at $\sqrt{5.02}$ \si{TeV} gave rise to several possible theoretical explanations with mutually exclusive assumptions about the state of matter created immediately after the collision.\\

% What was done?
The thesis at hand deployed the technique of two-particle correlation functions on an event sample biased towards hard events to gather further insight into the structure of the studied collisions. In its course, a novel approach to the correction of detector efficiencies was also deployed.

% What was found?
The flow-like excess in the measurements compared to DPMJET generated data was found to be independent of the hardness of the event sample within the margin of statistical and systematical uncertainties. The prominent double-ridge which was formerly yielded by subtracting the low multiplicity two-particle correlation function from the high multiplicity one was, however, significantly altered. This can be contributed to the enhancement of the di-jet structure in low multiplicity.

% What is its impact?
The \gls{cgc} and hydrodynamical model were expected to exhibit no dependence on the hardness of the underlying event while a dependency for \gls{cr} was anticipated. The here presented findings are suggestive towards the former two but it is not yet possible to discard \gls{cr} as the origin of the flow-like behavior.

% Where to go from here?
Further studies on the origin of the MC non-closure for hard event samples have to be conducted in order to significantly decrease the systematic uncertainty making the findings more conclusive. Furthermore, the deployment of a track clustering algorithm for the selection of hard events might improve the quality of the hardened event sample enhancing on the flow-like behavior. Furthermore, this thesis focused on the hardened event sample but similar studies on the softened sample would be the most important complement to these studies.

Even when not conclusively ruling out, or confirming one of the three proposed and discussed models explaining the double-ridge structure, this thesis still delivers valuable further insight into the properties of \gls{pPb} collisions, which is much needed in the further understanding of the observed flow-like effects.

%%% Local Variables: 
%%% mode: latex
%%% TeX-master: "main"
%%% End: 
